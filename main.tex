
%% bare_conf.tex
%% V1.3
%% 2007/01/11
%% by Michael Shell
%% See:
%% http://www.michaelshell.org/
%% for current contact information.
%%
%% This is a skeleton file demonstrating the use of IEEEtran.cls
%% (requires IEEEtran.cls version 1.7 or later) with an IEEE conference paper.
%%
%% Support sites:
%% http://www.michaelshell.org/tex/ieeetran/
%% http://www.ctan.org/tex-archive/macros/latex/contrib/IEEEtran/
%% and
%% http://www.ieee.org/

%%*************************************************************************
%% Legal Notice:
%% This code is offered as-is without any warranty either expressed or
%% implied; without even the implied warranty of MERCHANTABILITY or
%% FITNESS FOR A PARTICULAR PURPOSE! 
%% User assumes all risk.
%% In no event shall IEEE or any contributor to this code be liable for
%% any damages or losses, including, but not limited to, incidental,
%% consequential, or any other damages, resulting from the use or misuse
%% of any information contained here.
%%
%% All comments are the opinions of their respective authors and are not
%% necessarily endorsed by the IEEE.
%%
%% This work is distributed under the LaTeX Project Public License (LPPL)
%% ( http://www.latex-project.org/ ) version 1.3, and may be freely used,
%% distributed and modified. A copy of the LPPL, version 1.3, is included
%% in the base LaTeX documentation of all distributions of LaTeX released
%% 2003/12/01 or later.
%% Retain all contribution notices and credits.
%% ** Modified files should be clearly indicated as such, including  **
%% ** renaming them and changing author support contact information. **
%%
%% File list of work: IEEEtran.cls, IEEEtran_HOWTO.pdf, bare_adv.tex,
%%                    bare_conf.tex, bare_jrnl.tex, bare_jrnl_compsoc.tex
%%*************************************************************************

% *** Authors should verify (and, if needed, correct) their LaTeX system  ***
% *** with the testflow diagnostic prior to trusting their LaTeX platform ***
% *** with production work. IEEE's font choices can trigger bugs that do  ***
% *** not appear when using other class files.                            ***
% The testflow support page is at:
% http://www.michaelshell.org/tex/testflow/



% Note that the a4paper option is mainly intended so that authors in
% countries using A4 can easily print to A4 and see how their papers will
% look in print - the typesetting of the document will not typically be
% affected with changes in paper size (but the bottom and side margins will).
% Use the testflow package mentioned above to verify correct handling of
% both paper sizes by the user's LaTeX system.
%
% Also note that the "draftcls" or "draftclsnofoot", not "draft", option
% should be used if it is desired that the figures are to be displayed in
% draft mode.
%
\documentclass[conference]{IEEEtran}
\usepackage{blindtext, graphicx}
\usepackage{url}
% Add the compsoc option for Computer Society conferences.
%
% If IEEEtran.cls has not been installed into the LaTeX system files,
% manually specify the path to it like:
% \documentclass[conference]{../sty/IEEEtran}





% Some very useful LaTeX packages include:
% (uncomment the ones you want to load)


% *** MISC UTILITY PACKAGES ***
%
%\usepackage{ifpdf}
% Heiko Oberdiek's ifpdf.sty is very useful if you need conditional
% compilation based on whether the output is pdf or dvi.
% usage:
% \ifpdf
%   % pdf code
% \else
%   % dvi code
% \fi
% The latest version of ifpdf.sty can be obtained from:
% http://www.ctan.org/tex-archive/macros/latex/contrib/oberdiek/
% Also, note that IEEEtran.cls V1.7 and later provides a builtin
% \ifCLASSINFOpdf conditional that works the same way.
% When switching from latex to pdflatex and vice-versa, the compiler may
% have to be run twice to clear warning/error messages.






% *** CITATION PACKAGES ***
%
%\usepackage{cite}
% cite.sty was written by Donald Arseneau
% V1.6 and later of IEEEtran pre-defines the format of the cite.sty package
% \cite{} output to follow that of IEEE. Loading the cite package will
% result in citation numbers being automatically sorted and properly
% "compressed/ranged". e.g., [1], [9], [2], [7], [5], [6] without using
% cite.sty will become [1], [2], [5]--[7], [9] using cite.sty. cite.sty's
% \cite will automatically add leading space, if needed. Use cite.sty's
% noadjust option (cite.sty V3.8 and later) if you want to turn this off.
% cite.sty is already installed on most LaTeX systems. Be sure and use
% version 4.0 (2003-05-27) and later if using hyperref.sty. cite.sty does
% not currently provide for hyperlinked citations.
% The latest version can be obtained at:
% http://www.ctan.org/tex-archive/macros/latex/contrib/cite/
% The documentation is contained in the cite.sty file itself.






% *** GRAPHICS RELATED PACKAGES ***
%
\ifCLASSINFOpdf
  % \usepackage[pdftex]{graphicx}
  % declare the path(s) where your graphic files are
  % \graphicspath{{../pdf/}{../jpeg/}}
  % and their extensions so you won't have to specify these with
  % every instance of \includegraphics
  % \DeclareGraphicsExtensions{.pdf,.jpeg,.png}
\else
  % or other class option (dvipsone, dvipdf, if not using dvips). graphicx
  % will default to the driver specified in the system graphics.cfg if no
  % driver is specified.
  % \usepackage[dvips]{graphicx}
  % declare the path(s) where your graphic files are
  % \graphicspath{{../eps/}}
  % and their extensions so you won't have to specify these with
  % every instance of \includegraphics
  % \DeclareGraphicsExtensions{.eps}
\fi
% graphicx was written by David Carlisle and Sebastian Rahtz. It is
% required if you want graphics, photos, etc. graphicx.sty is already
% installed on most LaTeX systems. The latest version and documentation can
% be obtained at: 
% http://www.ctan.org/tex-archive/macros/latex/required/graphics/
% Another good source of documentation is "Using Imported Graphics in
% LaTeX2e" by Keith Reckdahl which can be found as epslatex.ps or
% epslatex.pdf at: http://www.ctan.org/tex-archive/info/
%
% latex, and pdflatex in dvi mode, support graphics in encapsulated
% postscript (.eps) format. pdflatex in pdf mode supports graphics
% in .pdf, .jpeg, .png and .mps (metapost) formats. Users should ensure
% that all non-photo figures use a vector format (.eps, .pdf, .mps) and
% not a bitmapped formats (.jpeg, .png). IEEE frowns on bitmapped formats
% which can result in "jaggedy"/blurry rendering of lines and letters as
% well as large increases in file sizes.
%
% You can find documentation about the pdfTeX application at:
% http://www.tug.org/applications/pdftex





% *** MATH PACKAGES ***
%
\usepackage[cmex10]{amsmath}
\usepackage{stmaryrd}
% A popular package from the American Mathematical Society that provides
% many useful and powerful commands for dealing with mathematics. If using
% it, be sure to load this package with the cmex10 option to ensure that
% only type 1 fonts will utilized at all point sizes. Without this option,
% it is possible that some math symbols, particularly those within
% footnotes, will be rendered in bitmap form which will result in a
% document that can not be IEEE Xplore compliant!
%
% Also, note that the amsmath package sets \interdisplaylinepenalty to 10000
% thus preventing page breaks from occurring within multiline equations. Use:
%\interdisplaylinepenalty=2500
% after loading amsmath to restore such page breaks as IEEEtran.cls normally
% does. amsmath.sty is already installed on most LaTeX systems. The latest
% version and documentation can be obtained at:
% http://www.ctan.org/tex-archive/macros/latex/required/amslatex/math/





% *** SPECIALIZED LIST PACKAGES ***
%
%\usepackage{algorithmic}
% algorithmic.sty was written by Peter Williams and Rogerio Brito.
% This package provides an algorithmic environment fo describing algorithms.
% You can use the algorithmic environment in-text or within a figure
% environment to provide for a floating algorithm. Do NOT use the algorithm
% floating environment provided by algorithm.sty (by the same authors) or
% algorithm2e.sty (by Christophe Fiorio) as IEEE does not use dedicated
% algorithm float types and packages that provide these will not provide
% correct IEEE style captions. The latest version and documentation of
% algorithmic.sty can be obtained at:
% http://www.ctan.org/tex-archive/macros/latex/contrib/algorithms/
% There is also a support site at:
% http://algorithms.berlios.de/index.html
% Also of interest may be the (relatively newer and more customizable)
% algorithmicx.sty package by Szasz Janos:
% http://www.ctan.org/tex-archive/macros/latex/contrib/algorithmicx/




% *** ALIGNMENT PACKAGES ***
%
\usepackage{array}
% Frank Mittelbach's and David Carlisle's array.sty patches and improves
% the standard LaTeX2e array and tabular environments to provide better
% appearance and additional user controls. As the default LaTeX2e table
% generation code is lacking to the point of almost being broken with
% respect to the quality of the end results, all users are strongly
% advised to use an enhanced (at the very least that provided by array.sty)
% set of table tools. array.sty is already installed on most systems. The
% latest version and documentation can be obtained at:
% http://www.ctan.org/tex-archive/macros/latex/required/tools/


%\usepackage{mdwmath}
%\usepackage{mdwtab}
% Also highly recommended is Mark Wooding's extremely powerful MDW tools,
% especially mdwmath.sty and mdwtab.sty which are used to format equations
% and tables, respectively. The MDWtools set is already installed on most
% LaTeX systems. The lastest version and documentation is available at:
% http://www.ctan.org/tex-archive/macros/latex/contrib/mdwtools/


% IEEEtran contains the IEEEeqnarray family of commands that can be used to
% generate multiline equations as well as matrices, tables, etc., of high
% quality.


%\usepackage{eqparbox}
% Also of notable interest is Scott Pakin's eqparbox package for creating
% (automatically sized) equal width boxes - aka "natural width parboxes".
% Available at:
% http://www.ctan.org/tex-archive/macros/latex/contrib/eqparbox/





% *** SUBFIGURE PACKAGES ***
%\usepackage[tight,footnotesize]{subfigure}
% subfigure.sty was written by Steven Douglas Cochran. This package makes it
% easy to put subfigures in your figures. e.g., "Figure 1a and 1b". For IEEE
% work, it is a good idea to load it with the tight package option to reduce
% the amount of white space around the subfigures. subfigure.sty is already
% installed on most LaTeX systems. The latest version and documentation can
% be obtained at:
% http://www.ctan.org/tex-archive/obsolete/macros/latex/contrib/subfigure/
% subfigure.sty has been superceeded by subfig.sty.



%\usepackage[caption=false]{caption}
%\usepackage[font=footnotesize]{subfig}
% subfig.sty, also written by Steven Douglas Cochran, is the modern
% replacement for subfigure.sty. However, subfig.sty requires and
% automatically loads Axel Sommerfeldt's caption.sty which will override
% IEEEtran.cls handling of captions and this will result in nonIEEE style
% figure/table captions. To prevent this problem, be sure and preload
% caption.sty with its "caption=false" package option. This is will preserve
% IEEEtran.cls handing of captions. Version 1.3 (2005/06/28) and later 
% (recommended due to many improvements over 1.2) of subfig.sty supports
% the caption=false option directly:
%\usepackage[caption=false,font=footnotesize]{subfig}
%
% The latest version and documentation can be obtained at:
% http://www.ctan.org/tex-archive/macros/latex/contrib/subfig/
% The latest version and documentation of caption.sty can be obtained at:
% http://www.ctan.org/tex-archive/macros/latex/contrib/caption/




% *** FLOAT PACKAGES ***
%
%\usepackage{fixltx2e}
% fixltx2e, the successor to the earlier fix2col.sty, was written by
% Frank Mittelbach and David Carlisle. This package corrects a few problems
% in the LaTeX2e kernel, the most notable of which is that in current
% LaTeX2e releases, the ordering of single and double column floats is not
% guaranteed to be preserved. Thus, an unpatched LaTeX2e can allow a
% single column figure to be placed prior to an earlier double column
% figure. The latest version and documentation can be found at:
% http://www.ctan.org/tex-archive/macros/latex/base/



%\usepackage{stfloats}
% stfloats.sty was written by Sigitas Tolusis. This package gives LaTeX2e
% the ability to do double column floats at the bottom of the page as well
% as the top. (e.g., "\begin{figure*}[!b]" is not normally possible in
% LaTeX2e). It also provides a command:
%\fnbelowfloat
% to enable the placement of footnotes below bottom floats (the standard
% LaTeX2e kernel puts them above bottom floats). This is an invasive package
% which rewrites many portions of the LaTeX2e float routines. It may not work
% with other packages that modify the LaTeX2e float routines. The latest
% version and documentation can be obtained at:
% http://www.ctan.org/tex-archive/macros/latex/contrib/sttools/
% Documentation is contained in the stfloats.sty comments as well as in the
% presfull.pdf file. Do not use the stfloats baselinefloat ability as IEEE
% does not allow \baselineskip to stretch. Authors submitting work to the
% IEEE should note that IEEE rarely uses double column equations and
% that authors should try to avoid such use. Do not be tempted to use the
% cuted.sty or midfloat.sty packages (also by Sigitas Tolusis) as IEEE does
% not format its papers in such ways.





% *** PDF, URL AND HYPERLINK PACKAGES ***
%
%\usepackage{url}
% url.sty was written by Donald Arseneau. It provides better support for
% handling and breaking URLs. url.sty is already installed on most LaTeX
% systems. The latest version can be obtained at:
% http://www.ctan.org/tex-archive/macros/latex/contrib/misc/
% Read the url.sty source comments for usage information. Basically,
% \url{my_url_here}.





% *** Do not adjust lengths that control margins, column widths, etc. ***
% *** Do not use packages that alter fonts (such as pslatex).         ***
% There should be no need to do such things with IEEEtran.cls V1.6 and later.
% (Unless specifically asked to do so by the journal or conference you plan
% to submit to, of course. )


% correct bad hyphenation here
\hyphenation{op-tical net-works semi-conduc-tor}




\usepackage{subfiles}
\usepackage{geometry}

% *** IMPORTS FOR PLUTUS LANGUAGE ***

\usepackage{bussproofs,amsmath,amssymb}



%\subfile{PlutusDefinitions}


% *** DEFINITIONS FOR PLUTUS LANGUAGE ***

\newcommand{\judgmentdef}[2]{\fbox{#1}

\vspace{0.5em}

#2}

\newcommand{\hyphen}{\operatorname{-}}
\newcommand{\apostrophe}{\operatorname{'}}
\newcommand{\vect}[2]{(#2)_{#1}}
\newcommand{\repetition}[1]{\overline{#1}}
\newcommand{\unrepetition}[1]{\underline{#1}}
\newcommand{\every}[2]{\forall #1(#2)}
\newcommand{\some}[2]{\exists #1(#2)}

\newcommand{\keyword}[1]{\mathtt{#1}}
\newcommand{\sizename}[1]{\$ #1}
\newcommand{\name}[1]{\mathtt{#1}}
%\newcommand{\qual}[2]{(\keyword{qual} ~ \name{#1} ~ \name{#2})}
\newcommand{\qual}[2]{#1\mathtt{.}#2}
%\newcommand{\qualcon}[2]{(\keyword{qualcon} ~ \name{#1} ~ \name{#2})}
\newcommand{\qualcon}[2]{#1\mathtt{.}#2}
\newcommand{\metatycon}{\kappa}
\newcommand{\metaqualtycon}{\hat{\kappa}}
\newcommand{\metatyname}{\nu}
\newcommand{\metaqualtyname}{\hat{\nu}}
\newcommand{\metacon}{c}
\newcommand{\metaqualcon}{\hat{c}}
\newcommand{\metaname}{n}
\newcommand{\metaqualname}{\hat{n}}
\newcommand{\variable}[1]{\mathtt{#1}}
\newcommand{\multiplearg}[2]{#1 \keyword{;} \mathtt{\ldots} \keyword{;} #2}
\newcommand{\listarg}[2]{#1 \keyword{,} \mathtt{\ldots} \keyword{,} #2}
\newcommand{\emptylist}{\keyword{\epsilon}}
\newcommand{\listargelement}[1]{\mathtt{\ldots} \keyword{,} #1 \keyword{,} \mathtt{\ldots}}
\newcommand{\altarg}[2]{#1 ~\keyword{|}~ \mathtt{\ldots} ~\keyword{|}~ #2}

\newcommand{\construct}[1]{\texttt{(} #1 \texttt{)}}
%\newcommand{\scope}[2]{(\keyword{scope} ~ (#1) ~ #2)}
%\newcommand{\decname}[1]{(\keyword{decname} ~ #1)}
\newcommand{\decname}[1]{\mathtt{#1}}
\newcommand{\ann}[2]{\construct{\keyword{isa} ~ #2 ~ #1}}
\newcommand{\letrec}[4]{\construct{\keyword{letrec} ~ #1 ~ #2 ~ #3 ~ #4}}
\newcommand{\lettype}[3]{\construct{\keyword{lettype} ~ #1 ~ #2 ~ #3}}
\newcommand{\sizeterm}[1]{\construct{\keyword{size} ~ #1}}
\newcommand{\abs}[2]{\construct{\keyword{abs} ~ #1 ~ #2}}
\newcommand{\inst}[2]{\construct{\keyword{inst} ~ #1 ~ #2}}
\newcommand{\lam}[2]{\construct{\keyword{lam} ~ #1 ~ #2}}
%\newcommand{\app}[2]{(\keyword{app} ~ #1 ~ #2)}
\newcommand{\app}[2]{\texttt{[} #1 ~ #2 \texttt{]}}
\newcommand{\fix}[2]{\construct{\keyword{fix} ~ #1 ~ #2}}
\newcommand{\con}[2]{\construct{\keyword{con} ~ #1 ~ #2}}
\newcommand{\conNullary}[1]{\construct{\keyword{con} ~ #1}}
\newcommand{\case}[2]{\construct{\keyword{case} ~ #1 ~ #2}}
\newcommand{\cl}[3]{\construct{#1 ~ \texttt{(} #2 \texttt{)} ~ #3}}
\newcommand{\varpat}{\keyword{varpat}}
\newcommand{\conpat}[1]{\construct{\keyword{conpat} ~ #1}}
\newcommand{\success}[1]{\construct{\keyword{success} ~ #1}}
\newcommand{\failure}{\construct{\keyword{failure}}}
\newcommand{\compbuiltin}[1]{\construct{\keyword{compbuiltin} ~ #1}}
\newcommand{\txhash}{\construct{\keyword{txhash}}}
\newcommand{\blocknum}{\construct{\keyword{blocknum}}}
\newcommand{\blocktime}{\construct{\keyword{blocktime}}}
\newcommand{\bind}[3]{\construct{\keyword{bind} ~ #1 ~ #2 ~ #3}}
\newcommand{\builtin}[1]{\construct{\keyword{builtin} ~ #1}}
\newcommand{\prim}[1]{\construct{\keyword{prim} ~ #1}}
%\newcommand{\primInt}[1]{(\keyword{primInteger} ~ #1)}
\newcommand{\primInt}[1]{\mathtt{#1}}
%\newcommand{\primFloat}[1]{(\keyword{primFloat} ~ #1)}
\newcommand{\primFloat}[1]{\mathtt{#1}}
%\newcommand{\primByteString}[1]{(\keyword{primByteString} ~ #1)}
\newcommand{\primByteString}[1]{\mathtt{#1}}
\newcommand{\isFun}[1]{\construct{\keyword{isFun} ~ #1}}
\newcommand{\isCon}[1]{\construct{\keyword{isCon} ~ #1}}
\newcommand{\isConName}[2]{\construct{\keyword{isConName} ~ #1 ~ #2}}
\newcommand{\isInt}[1]{\construct{\keyword{isInteger} ~ #1}}
\newcommand{\isFloat}[1]{\construct{\keyword{isFloat} ~ #1}}
\newcommand{\isByteString}[1]{\construct{\keyword{isByteString} ~ #1}}
\newcommand{\prg}[1]{\construct{\keyword{program} ~ #1}}
\newcommand{\mdle}[4]{\construct{\keyword{module} ~ #1 ~ #2 ~ #3 ~ #4}}
\newcommand{\impd}[1]{\construct{\keyword{import} ~ #1}}
\newcommand{\expd}[2]{\construct{\keyword{export} ~ \construct{#1} ~ \construct{#2}}}
\newcommand{\dataexport}[2]{\construct{#1 ~ \construct{#2}}}
\newcommand{\typeexp}[2]{\construct{#1 ~ \construct{#2}}}
\newcommand{\datadecl}[3]{\construct{\keyword{data} ~ #1 ~ \construct{#2} ~ #3}}
\newcommand{\typedecl}[2]{\construct{\keyword{type} ~ #1 ~ #2}}
\newcommand{\termdecl}[2]{\construct{\keyword{declare} ~ #1 ~ #2}}
%\newcommand{\expt}[2]{(\keyword{exp} ~ \name{#1} ~ #2)}
%\newcommand{\loc}[2]{(\keyword{loc} ~ \name{#1} ~ #2)}
%\newcommand{\exptcon}[2]{(\keyword{expcon} ~ \name{#1} ~ #2)}
%\newcommand{\loccon}[2]{(\keyword{loccon} ~ \name{#1} ~ #2)}
\newcommand{\defdecl}[2]{\construct{\keyword{define} ~ #1 ~ #2}}
\newcommand{\funT}[2]{\construct{\keyword{fun} ~ #1 ~ #2}}
\newcommand{\fixT}[3]{\construct{\keyword{fix} ~ #1 ~ #2 ~ #3}}
\newcommand{\conT}[2]{\construct{\keyword{con} ~ #1 ~ #2}}
\newcommand{\conTNullary}[1]{\construct{\keyword{con} ~ #1}}
\newcommand{\compT}[1]{\construct{\keyword{comp} ~ #1}}
\newcommand{\forallT}[3]{\construct{\keyword{forall} ~ #1 ~ #2 ~ #3}}
\newcommand{\bytestringT}[1]{\construct{\keyword{bytestring} ~ #1}}
\newcommand{\integerT}[1]{\construct{\keyword{integer} ~ #1}}
\newcommand{\floatT}[1]{\construct{\keyword{float} ~ #1}}
\newcommand{\sizeT}[1]{\construct{\keyword{size} ~ #1}}
\newcommand{\lamT}[3]{\construct{\keyword{lam} ~ #1 ~ #2 ~ #3}}
\newcommand{\appT}[2]{\texttt{[} #1 ~ #2 \texttt{]}}
\newcommand{\typeK}{\construct{\keyword{type}}}
\newcommand{\funK}[2]{\construct{\keyword{fun} ~ #1 ~ #2}}
\newcommand{\sizeK}{\construct{\keyword{size}}}
\newcommand{\sizelit}[1]{\$ #1}
\newcommand{\conaritydec}[3]{\construct{\keyword{conarity} ~ #1 ~ \construct{#2} ~ \construct{#3}}}
\newcommand{\conarity}[3]{\construct{\construct{#1} ~ \construct{#2} ~ #3}}
\newcommand{\tyconaritydec}[2]{\construct{\keyword{tyconarity} ~ #1 ~ \construct{#2}}}
\newcommand{\tyconarity}[1]{\construct{#1}}
\newcommand{\typesignature}[2]{\construct{#1 ~ #2}}
\newcommand{\kindsignature}[2]{\construct{#1 ~ #2}}

\newcommand{\checkJ}[2]{#1 \ni #2}
\newcommand{\synthJ}[2]{#1 \in #2}
\newcommand{\synthsplittopJ}[2]{#1 \vdash #2}
\newcommand{\synthsplitbottomJ}[1]{\in #1}
\newcommand{\builtinSig}[5]{#1 ~ \textrm{is} ~ \forallT{#2}{#3}{\funT{#4}{#5}}}
\newcommand{\compbuiltinSig}[2]{#1 ~ \textrm{is} ~ #2}
\newcommand{\subtypeJ}[2]{#1 ~ \sqsubseteq ~ #2}
\newcommand{\clause}[4]{#1 ; #2 \vdash ~ #3 ~ \mathrm{clause} ~ #4}
%\newcommand{\decl}[4]{#1 ~ \vdash ~ #2 ~ decl ~ \name{#3} ~ \dashv ~ #4}
\newcommand{\edecl}[4]{#1 ~ \vdash ~ #2 ~ \mathrm{edecl} ~ #3 ~ \dashv ~ #4}
\newcommand{\edeclsplit}[4]{\begin{array}{ll}#1 ~ \vdash ~ #2 ~ \mathrm{edecl} ~ #3\\\qquad\dashv ~ #4\end{array}}
\newcommand{\ldecl}[4]{#1 ~ \vdash ~ #2 ~ \mathrm{ldecl} ~ #3 ~ \dashv ~ #4}
\newcommand{\ldeclsplit}[4]{\begin{array}{ll}#1 ~ \vdash ~ #2 ~ \mathrm{ldecl} ~ #3\\\qquad\dashv ~ #4\end{array}}
\newcommand{\defs}[3]{#1 ~ \vdash ~ #2 ~ \mathrm{defs} ~ #3}

\newcommand{\modJ}[1]{\mathrm{mod} ~ #1}
\newcommand{\exportedtypeJ}[2]{\mathrm{exptype} ~ #1.#2}
\newcommand{\exportedtermJ}[2]{\mathrm{expterm} ~ #1.#2}
\newcommand{\termnameJ}[2]{#1 : #2}
\newcommand{\defJ}[2]{#1 = #2}
\newcommand{\typenameJ}[3]{#1 = #2 :: #3}
\newcommand{\tyconJ}[2]{#1 :: \funK{#2}{\typeK{}}}
\newcommand{\conJ}[5]{#1 : \forallT{#2}{#3}{\funT{#4}{\appT{#5}{#2}}}}
\newcommand{\ctxni}[2]{#1 \ni #2}
\newcommand{\ctxnotni}[2]{#1 \not\ni #2}
\newcommand{\permits}[3]{#1 \vdash #2 : #3}
\newcommand{\permitstype}[3]{#1 \vdash #2 :: #3}
\newcommand{\permitscon}[6]{#1 \vdash #2 : \forallT{#3}{#4}{\funT{#5}{\appT{#6}{#3}}}}
\newcommand{\permitstycon}[3]{#1 \vdash #2 :: \funK{#3}{\typeK{}}}
\newcommand{\exportsToNominalContext}[2]{\lfloor #2 \rfloor_{#1}}
\newcommand{\exportsToNominalContextType}[2]{\lfloor #2 \rfloor^{T}_{#1}}
\newcommand{\exportsToNominalContextAlt}[2]{\lfloor #2 \rfloor^{alt}_{#1}}
\newcommand{\exportsToNominalContextTerm}[2]{\lfloor #2 \rfloor^{M}_{#1}}

\newcommand{\hypJ}[2]{#1 \vdash #2}
\newcommand{\termJ}[2]{#1 : #2}
\newcommand{\typeJ}[2]{#1 :: #2}
\newcommand{\istypeJ}[2]{#1 :: #2}
\newcommand{\unifiesJ}[3]{#1 ~ / ~ #2 ~ \triangleright ~ #3}
\newcommand{\extractsJ}[4]{#1 ~ \mathrm{on} ~ #2 ~ \mathrm{extracts} ~ #3 ~ \triangleright ~ #4}
\newcommand{\clauseJ}[3]{#1 \ni #2 ~ \mathrm{clause} \in #3}
\newcommand{\hypclauseJ}[4]{#1 ; #2 \vdash #3 \ni #4}
\newcommand{\programJ}[1]{\vdash #1}
\newcommand{\moduleJ}[2]{#1 \vdash #2}
%\newcommand{\moduleJsplit}[3]{\begin{array}{ll}#1 ~ ~ \vdash ~ #2 ~ \mathrm{module}\\\qquad\dashv #3\end{array}}
\newcommand{\modulesplitJ}[7]{\begin{array}{c}#1 ~ \vdash ~ \keyword{(module} ~ #2 ~ #3 ~ #4 \\ #5 ~ #6 \keyword{)} ~ \mathrm{module} ~ \dashv #7\end{array}}
\newcommand{\declJ}[2]{#1 \vdash #2}
\newcommand{\kindsig}[2]{\construct{#1 ~ #2}}
\newcommand{\alt}[2]{\construct{#1 ~ #2}}
\newcommand{\altNullary}[1]{\construct{#1}}
\newcommand{\exptdeclJ}[4]{#1 ~ \vdash ~ #2 ~ \mathrm{expdecl} ~ #3 ~ \dashv ~ #4}
\newcommand{\locdeclJ}[4]{#1 ~ \vdash ~ #2 ~ \mathrm{locdecl} ~ #3 ~ \dashv ~ #4}
\newcommand{\exptaltJ}[6]{#1 ~ \vdash ~ #2 ~ \mathrm{expalt} ~ #3 ~ \mathrm{on} ~ #4 ~ \mathrm{in} ~ #5 ~ \dashv ~ #6}
\newcommand{\exptaltsplitJ}[6]{\begin{array}{c}#1 ~ \vdash ~ #2 ~ \mathrm{expalt} ~ #3 ~ \mathrm{on} ~ #4 ~ \mathrm{in} ~ #5\\\dashv ~ #6\end{array}}
\newcommand{\localtsplitJ}[6]{\begin{array}{c}#1 ~ \vdash ~ #2 ~ \mathrm{localt} ~ #3 ~ \mathrm{on} ~ #4 ~ \mathrm{in} ~ #5\\\dashv ~ #6\end{array}}
\newcommand{\altsplitJ}[6]{\begin{array}{c}#1 ~ \vdash ~ #2 ~ \mathrm{alt} ~ #3 ~ \mathrm{on} ~ #4 ~ \mathrm{in} ~ #5\\\dashv ~ #6\end{array}}
\newcommand{\localtJ}[6]{#1 ~ \vdash ~ #2 ~ \mathrm{localt} ~ #3 ~ \mathrm{on} ~ #4 ~ \mathrm{in} ~ #5 ~ \dashv ~ #6}
\newcommand{\altJ}[2]{#1 \vdash #2 ~ \mathrm{alt}}
\newcommand{\defineJ}[4]{#1 ~ \vdash ~ #2 ~ \mathrm{define} ~ #3 ~ \dashv ~ #4}
\newcommand{\valJ}[1]{#1 ~ \mathrm{val}}
\newcommand{\tyvalJ}[1]{#1 ~ \mathrm{tyval}}

\newcommand{\declenv}[1]{\lfloor{}#1\rfloor{}}

\newcommand{\decgen}[2]{#1 ~>\!\!>\!\!>~ #2}
\newcommand{\decgenmodule}[2]{#1 ~>\!\!>~ #2}
\newcommand{\ok}[1]{\texttt{(} \keyword{ok} ~ #1 \texttt{)}}
\newcommand{\err}{\keyword{err}}
\newcommand{\hole}{\circ}
\newcommand{\inAppTLeft}[2]{\appT{#1}{#2}}
\newcommand{\inAppTRight}[2]{\appT{#1}{#2}}
\newcommand{\inFunTLeft}[2]{\funT{#1}{#2}}
\newcommand{\inFunTRight}[2]{\funT{#1}{#2}}
\newcommand{\inConT}[4]{\conT{#1}{#2 ~ #3 ~ #4}}
\newcommand{\inForallT}[3]{\forallT{#1}{#2}{#3}}
\newcommand{\inCompT}[1]{\compT{#1}}
\newcommand{\typeReduces}[2]{#1 ~ \Rightarrow_{ty} ~ #2}
\newcommand{\typeStep}[2]{#1 ~ \rightarrow_{ty} ~ #2}
\newcommand{\typeMultistep}[2]{#1 ~ \rightarrow_{ty}^* ~ #2}
\newcommand{\inAnnLeft}[2]{\ann{#1}{#2}}
\newcommand{\inInstLeft}[2]{\inst{#1}{#2}}
\newcommand{\inAppLeft}[2]{\app{#1}{#2}}
\newcommand{\inAppRight}[2]{\app{#1}{#2}}
\newcommand{\inCon}[2]{\con{#1}{#2}}
\newcommand{\inCase}[2]{\case{#1}{#2}}
\newcommand{\caseValue}[2]{\mathrm{case\ on} ~ #1 ~ \mathrm{of} ~ #2}
\newcommand{\inSuccess}[1]{\success{#1}}
\newcommand{\inBind}[3]{\bind{#1}{#2}{#3}}
\newcommand{\inBuiltin}[2]{\builtin{#1}{#2}}
\newcommand{\ctxsubst}[2]{#1\{#2\}}
\newcommand{\reduces}[3]{#2 ~ \Rightarrow_{#1} ~ #3}
\newcommand{\builtinReduces}[3]{#1 ~ \mathrm{on} ~ #2 ~ \mathrm{reduces ~ to} ~ #3}
\newcommand{\builtinValue}[2]{#1 ~ \mathrm{on} ~ #2}
\newcommand{\step}[3]{#2 ~ \rightarrow_{#1} ~ #3}
\newcommand{\multistep}[3]{#2 ~ \rightarrow^*_{#1} ~ #3}
\newcommand{\multistepIndexed}[4]{#2 ~ \rightarrow^{#3}_{#1} ~ #4}
\newcommand{\normalform}[1]{\llbracket #1 \rrbracket}
\newcommand{\evalprog}[2]{#1 ~\Longrightarrow~ #2}
\newcommand{\evalsto}[3]{#2 ~\Downarrow_{#1}~ #3}
\newcommand{\builtineval}[3]{\mathrm{built-in} ~ \name{#1} ~ \mathrm{evals ~ on} ~ #2 ~ \mathrm{to} ~ #3}
\newcommand{\matches}[4]{#1 , #2 ~\sim~ #3 ~\triangleright~ #4}
\newcommand{\subst}[3]{[#1/#2]#3}
\newcommand{\substmulti}[2]{[#1]#2}
\newcommand{\executes}[3]{#2 ~\rightsquigarrow^*_{#1}~ #3}
\newcommand{\executesIndexed}[4]{#2 ~\rightsquigarrow^{#3}_{#1}~ #4}
\newcommand{\cksteps}[3]{#2 ~\mapsto_{#1}~ #3}
\newcommand{\inAnnLeftFrame}[1]{\ann{\_}{#1}}
\newcommand{\inInstLeftFrame}[1]{\inst{\_}{#1}}
\newcommand{\inAppLeftFrame}[1]{\app{\_}{#1}}
\newcommand{\inAppRightFrame}[1]{\app{#1}{\_}}
\newcommand{\inConFrame}[3]{\con{#1}{#2 ~ \_ ~ #3}}
\newcommand{\inCaseFrame}[1]{\case{\_}{#1}}
\newcommand{\inSuccessFrame}{\success{\_}}
\newcommand{\inBindFrame}[2]{\bind{\_}{#1}{#2}}
\newcommand{\inBuiltinFrame}[3]{\builtin{#1}{#2 ~ \_ ~ #3}}
\newcommand{\ckforward}[2]{#1 \triangleright #2}
\newcommand{\ckbackward}[2]{#1 \triangleleft #2}

\newcommand{\sldecname}[1]{\name{#1}}
\newcommand{\slann}[2]{#1 ~\keyword{:}~ #2}
\newcommand{\sllet}[2]{\keyword{let} ~ \keyword{\{} ~ #1 ~ \keyword{\} ~ in} ~ #2}
\newcommand{\slletdec}[3]{\variable{#1}  ~ \keyword{:}  ~ #2  ~ \keyword{\{} ~  #3 ~ \keyword{\}}}
\newcommand{\sllam}[2]{\keyword{\lambda} \variable{#1}  ~ \keyword{\to}  ~ #2}
\newcommand{\slapp}[2]{#1  ~ #2}
\newcommand{\slcon}[2]{\name{#1}  ~ #2}
\newcommand{\slconnullary}[1]{\name{#1}}
\newcommand{\slcase}[2]{\keyword{case}  ~ #1  ~ \keyword{of  ~ \{}  ~ #2  ~ \keyword{\}}}
\newcommand{\slsuccess}[1]{\keyword{success} ~ #1}
\newcommand{\slfailure}{\keyword{failure}}
\newcommand{\slbind}[3]{\keyword{do  ~ \{}  ~ #1 ~ \keyword{\leftarrow} ~ #2  ~ \keyword{;} ~ #3 ~ \keyword{\}}}
\newcommand{\slbuiltin}[2]{\keyword{!}\name{#1} ~ #2}
\newcommand{\slcl}[2]{#1  ~ \keyword{\to}  ~ #2}
\newcommand{\sltydec}[2]{\keyword{data}  ~ #1  ~ \keyword{=}  ~ #2}
\newcommand{\sltycon}[2]{\name{#1} ~ #2}
\newcommand{\sltyconnullary}[1]{\name{#1}}
\newcommand{\slarr}[2]{#1  ~ \keyword{\to}  ~ #2}
\newcommand{\slforall}[2]{\keyword{\forall} #1\keyword{.} ~ #2}
\newcommand{\slcomp}[1]{\name{Comp} ~ #1}
\newcommand{\sltmdec}[3]{\name{#1}  ~ \keyword{:}  ~ #2  ~ \keyword{\{} ~  #3 ~ \keyword{\}}}
\newcommand{\sltmdecclauses}[3]{\name{#1}  ~ \keyword{:}  ~ #2  ~ #3}
\newcommand{\sltmcl}[3]{\name{#1} ~ #2 ~ \keyword{=} ~ #3}

\newcommand{\tyconsig}[1]{\keyword{\star}^#1}
\newcommand{\consig}[3]{\keyword{[} \variable{#1} \keyword{](} #2 \keyword{)} #3}
\newcommand{\elabprog}[5]{#1;#2 ~\vdash~ #3 ~\dashv~ #4;#5}
\newcommand{\elabtydec}[4]{#1 ~\vdash~ \textit{type}  ~ #2  ~ \textit{alts}  ~ #3 ~\dashv~ #4}
\newcommand{\elabalt}[4]{#1 ~\vdash~ #2  ~ \textit{alt}  ~ #3 ~\dashv~ #4}
\newcommand{\elabaltsplit}[4]{\begin{array}{c}#1 ~\vdash~ #2  ~ \textit{alt}  ~ #3\\\dashv~ #4\end{array}}
\newcommand{\elabtmdec}[6]{#1;#2 ~\vdash~ \textit{term}  ~ \name{#3}  ~ \textit{type} ~ #4  ~ \textit{def}  ~ #5 ~\dashv~ #6}
\newcommand{\elabtmdecpartial}[5]{#1 ~\vdash~ \textit{term}  ~ \name{#2}  ~ \textit{type} ~ #3  ~ \textit{def}  ~ #4 ~\dashv~ #5}
\newcommand{\chkfull}[6]{#1 ~;~ #2 ~;~ #3 ~\vdash~ #4 ~\ni~ #5 ~\triangleright~ #6}
\newcommand{\synfull}[6]{#1 ~;~ #2 ~;~ #3 ~\vdash~ #4 ~\triangleright~ #5 ~\in~ #6}
\newcommand{\chkpartial}[4]{#1 ~\vdash~ #2 ~\ni~ #3 ~\triangleright~ #4}
\newcommand{\synpartial}[4]{#1 ~\vdash~ #2 ~\triangleright~ #3 ~\in~ #4}
\newcommand{\chk}[3]{#1 ~\ni~ #2 ~\triangleright~ #3}
\newcommand{\syn}[3]{#1 ~\triangleright~ #2 ~\in~ #3}
\newcommand{\synsplit}[3]{\begin{array}{c}#1 ~\triangleright\\#2 ~\in~ #3\end{array}}
%\newcommand{\ctxni}[3]{#1 ~\ni~ \variable{#2} ~:~ #3}
\newcommand{\declni}[3]{#1 ~\ni~ \name{#2} ~:~ #3}
\newcommand{\signi}[3]{#1 ~\ni~ \name{#2} ~:~ #3}
\newcommand{\isty}[1]{#1  ~ type}
\newcommand{\chktypartial}[2]{#1 ~\vdash~ \isty{#2}}
\newcommand{\subtype}[2]{#1 ~\sqsubseteq~ #2}
\newcommand{\chkclause}[6]{\slcl{#1}{#2} ~\triangleright~ \slcl{#3}{#4} ~ \textit{from}  ~ #5 ~ \textit{to} ~ #6}
\newcommand{\chkclausefull}[9]{#1;#2;#3 ~\vdash~ \chkclause{#4}{#5}{#6}{#7}{#8}{#9}}
\newcommand{\chkclausefullsplit}[9]{\begin{array}{c}#1;#2;#3 ~\vdash~ \slcl{#4}{#5} ~\triangleright\\ \slcl{#6}{#7} ~ \textit{from}  ~ #8 ~ \textit{to} ~ #9\end{array}}
\newcommand{\chktyfull}[3]{#1;#2 ~\vdash~ \isty{#3}}
\newcommand{\chkpatfull}[5]{#1 ~\vdash~ #2  ~ \textit{pattern} ~ #3 ~\triangleright~ #4 ~\dashv~ #5}
\newcommand{\chkpatfullsplit}[5]{\begin{array}{c}#1 ~\vdash~ #2  ~ \textit{pattern} ~ #3 ~\triangleright\\ #4 ~\dashv~ #5\end{array}}





\begin{document}
%
% paper title
% can use linebreaks \\ within to get better formatting as desired
\title{Formal Specification of\\the Plutus Core Language}


% author names and affiliations
% use a multiple column layout for up to three different
% affiliations



%\author{\IEEEauthorblockN{Darryl McAdams}
%\IEEEauthorblockA{Email: darryl.mcadams@iohk.io\\
%Slack: @darryl.mcadams}



%\and
%\IEEEauthorblockN{Homer Simpson}
%\IEEEauthorblockA{Twentieth Century Fox\\
%Springfield, USA\\
%Email: homer@thesimpsons.com}


% conference papers do not typically use \thanks and this command
% is locked out in conference mode. If really needed, such as for
% the acknowledgment of grants, issue a \IEEEoverridecommandlockouts
% after \documentclass

% for over three affiliations, or if they all won't fit within the width
% of the page, use this alternative format:
% 
%\author{\IEEEauthorblockN{Michael Shell\IEEEauthorrefmark{1},
%Homer Simpson\IEEEauthorrefmark{2},
%James Kirk\IEEEauthorrefmark{3}, 
%Montgomery Scott\IEEEauthorrefmark{3} and
%Eldon Tyrell\IEEEauthorrefmark{4}}
%\IEEEauthorblockA{\IEEEauthorrefmark{1}School of Electrical and Computer Engineering\\
%Georgia Institute of Technology,
%Atlanta, Georgia 30332--0250\\ Email: see http://www.michaelshell.org/contact.html}
%\IEEEauthorblockA{\IEEEauthorrefmark{2}Twentieth Century Fox, Springfield, USA\\
%Email: homer@thesimpsons.com}
%\IEEEauthorblockA{\IEEEauthorrefmark{3}Starfleet Academy, San Francisco, California 96678-2391\\
%Telephone: (800) 555--1212, Fax: (888) 555--1212}
%\IEEEauthorblockA{\IEEEauthorrefmark{4}Tyrell Inc., 123 Replicant Street, Los Angeles, California 90210--4321}}




% use for special paper notices
%\IEEEspecialpapernotice{(Invited Paper)}




% make the title area
\maketitle


%\begin{abstract}
%\boldmath
%The Plutus Language is outlined, together with the major
%design decisions for implementations. A formal specification of the
%language is given, including an elaborator and bidirectional type
%system.
%\end{abstract}
% IEEEtran.cls defaults to using nonbold math in the Abstract.
% This preserves the distinction between vectors and scalars. However,
% if the journal you are submitting to favors bold math in the abstract,
% then you can use LaTeX's standard command \boldmath at the very start
% of the abstract to achieve this. Many IEEE journals frown on math
% in the abstract anyway.

% Note that keywords are not normally used for peerreview papers.
%\begin{IEEEkeywords}
%IEEEtran, journal, \LaTeX, paper, template.
%\end{IEEEkeywords}






% For peer review papers, you can put extra information on the cover
% page as needed:
% \ifCLASSOPTIONpeerreview
% \begin{center} \bfseries EDICS Category: 3-BBND \end{center}
% \fi
%
% For peerreview papers, this IEEEtran command inserts a page break and
% creates the second title. It will be ignored for other modes.
\IEEEpeerreviewmaketitle




% \section{Plutus Core}

% Plutus Core is a typed, strict, eagerly-reduced $\lambda$-calculus design to run as a transaction validation scripting language on blockchain systems. It's designed to be simple and easy to reason about using mechanized proof assistants and automated theorem provers. The grammar of the language is given in Figures \ref{fig:Plutus_core_grammar} and \ref{fig:Plutus_core_lexical_grammar}, using a modified s-expression format. As is standard in $\lambda$-calculi, we have variables, $\lambda$-abstractions, and application. In addition to this, there are also polymorphic and instantiation, data constructors, case expressions, declared names, computational primitives, primitive values, and built-in functions. Terms live within top-level declarations, which can also consist of data and type declarations as well as type signature declarations. Declarations themselves reside within modules, and a program is a collection of such modules.



\subfile{figures/PlutusCoreLexicalGrammar}

\subfile{figures/PlutusCoreGrammar}




% In this grammar, we have multi-argument application, both in types (\(\appT{T}{T^+}\)) and in terms (\(\app{M}{M^+}\)). This is to be understood as a convenient form of syntactic sugar for iterated binary application associated to the left, and the formal rules treat only the binary case.

% As an example, consider the program in Figure \ref{fig:Plutus_core_example}, which defines the type of natural numbers as well as lists, and the factorial and map functions. This program is not the most readable, which is to be expected from a representation intended for machine interpretation rather than human interpretation, but it does make explicit precisely what the roles are of the various parts.




%\subfile{figures/PlutusCoreExample}




% !!!! Example




% \section{Type Correctness}

%We define for Plutus Core a number of typing judgments which explain ways that a program can be well-formed. First, in Figure \ref{fig:Plutus_core_contexts}, we define the grammar of the various kinds of contexts that these judgments hold under. Nominal contexts contain information about the various declared names that exist within the system --- module names, exported types, exported terms, and then term names and their definitions, term constructors, type constructors, and type names and their definitions.

%We refer to the components of a nominal context by dotted names ($\Theta.\repetition{l}$, $\Theta.\repetition{\metaqualtycon{}|\metaqualtyname{}}$, $\Theta.\repetition{\metaqualcon{}|\metaqualname{}}$, and $\Theta.\repetition{nj}$). Variable contexts contain information about the nature of variables --- type variables with their kind, and term variables with their type. The overall context $\Theta$ consists of nominal and variable contexts, along with the name of the current module being elaborated and the name of imported modules. As with nominal contexts, we also refer to these by dotted names ($\Theta.l$, $\Theta.\repetition{l}$, $\Theta.\Delta$, and $\Theta.\Gamma$) In the inference rules, we use \(\Theta, \typeJ{\alpha}{K}\) to mean $\Theta$ with it's variable context $\Gamma$ extended with $\typeJ{\alpha}{K}$, and \(\Theta, \termJ{x}{A}\) to mean $\Theta$ with it's variable context $\Gamma$ extended with $\termJ{x}{A}$.

%Then, in Figure \ref{fig:Plutus_core_type_well_formedness}, we define what it means for a type construct to inhabit a kind. Plutus Core is a higher-kinded version of System-F with constructors and some primitive types, so we have a number of standard System-F rules together with some obvious extensions

%Next, in Figure \ref{fig:Plutus_core_type_checking}, we define the type checking judgment that explains when a type contains a term. This is defined together with Figure \ref{fig:Plutus_core_type_synthesis}'s type synthesis judgment, which explains how a term synthesizes a type. Together, these two judgments constitute a standard bidirectional type theory \cite{pfenning-bidi} \cite{christiansen-bidi}.

%A number of auxiliary judgments are defined in Figure \ref{fig:Plutus_core_auxiliary_judgments}. In particular, we define when a type contains a clause for that type's constructors, and how that then synthesizes a type. We also define what it means for a qualified name to be permitted for use. Finally, we define the various elaboration judgments in Figure \ref{fig:Plutus_core_elaboration_judgments}, which explain how declarations, modules, and programs elaborate out to complete nominal contexts. Declarations for the types and definitions of terms are separated into two distinct forms, rather than a single construct. One reason for this is that it makes type checking mutual recursion relatively simple, because all the types of names can be put before use sites.

%Finally, type synthesis for built-in operations (\(\builtinSig{n}{\repetition{S}}{T}\)) is given in tabular form rather than in inference rule form, in Figure \ref{fig:Plutus_core_builtins}, which also gives the reduction semantics. The types in the arguments column constitute $\repetition{T}$, and the type in the return column constitutes $T'$, in the judgment form. The same is done for synthesis of built-in computations (\(\compbuiltinSig{n}{T}\)).






\subfile{figures/PlutusCoreTypeCorrectness}









%\section{Reduction and Execution}




%The execution of a program in Plutus Core does not in itself result in any reduction. Instead, the declarations are bound to their appropriate names in a declaration environment $\delta$, which we will represent by a list of items of the form \(\metaqualname{} \mapsto M\). Then, designated names can be chosen to be reduced in this declaration environment generated. For instance, we might designate the name $\name{main}$ to be the name who's definition we reduce, as is done in Haskell.




%To give the computation rules for Plutus Core, we must define what the return values are of the language, as given in Figure \ref{fig:Plutus_core_return_values}. Rather than using values directly, we wrap them in a return value form, because reduction steps can fail. These then let us define a parameterized binary relation \(\multistep{\delta}{M}{R}\) which means $M$ eagerly reduces to $R$ using declarations $\delta$, in Figure \ref{fig:Plutus_core_reduction}. This uses a standard contextual dynamics to separate the local reductions, reduction contexts, and repeated reductions into separate judgments. We also define a step-indexed dynamics \(\multistepIndexed{\delta}{M}{n}{R}\), which means that $M$ reduces to $R$ using $\delta$ in at most $n$ steps. Step-indexed reduction is useful in settings where we want to limit the number of computational steps that can occur. These relations represent the transitive closure of the single-step reduction relation $\step{\delta}{M}{R}$, which is itself the lifting of local (i.e. $\beta$) reduction $\reduces{\delta}{M}{R}$ to the non-local setting by digging through a reduction context.

%One such setting for step-indexing is that of blockchain transactions, for which Plutus Core has been explicitly designed. In order to prevent transaction validation from looping indefinitely, or from simply taking an inordinate amount of time, which would be a serious security flaw in the blockchain systemn, we can use step indexing to put an upper bound on the number of computational steps that a program can have. In this setting, we would pick some upper bound $\mathit{max}$ and then perform reductions of terms $M$ by computing which $R$ is such that \(\multistepIndexed{\delta}{M}{\mathit{max}}{R}\).

%Because built-in reduction is implemented directly in terms of meta-language functionality, the specifications for them are subtly different than for other parts of this spec. In particular, we must explain what these meta-language implementations are that constitute the implicit spec. Primitive numeric integers are implemented as Haskell $\mathit{Integer}$s, primitive floats as $\mathit{Float}$, and primitive bytestrings as $\mathit{ByteString}$. For numeric built-ins, the operations are interpreted as the corresponding Haskell operations. So for example, $\mathit{addInteger}$ is interpreted as \((+) :: Integer \rightarrow Integer \rightarrow Integer\). The function names in the definitions are the same as the Haskell implementations where applicable. Some minor differences exist in some places, however. The cryptographic functions $\mathit{sha2\_256}$ and $\mathit{sha3\_256}$, in particular. They are implemented in terms of hashing into the $\mathit{SHA256}$ and $\mathit{SHA3256}$ digest types using the $\mathit{Crypto.Hash}$ and $\mathit{Crypto.Sign.Ed25519}$ modules. More indirectly, the specification for these are the cryptographic standards for SHA2 256 and SHA3 256.

%All of these operations are given in tabular form. The arguments column specifies what sorts of arguments are required for correct application of the given built in, which results in the production of an $\ok{V}$ return value that wraps the value given in the result column. When the arguments are not of the specified form, the result of the built in application is $\err{}$.

%A final note on built-in reduction is that some built-ins return constructed data using qualified names in the $\mathit{Prelude}$ module. This specification assumes that an implementation will have such a module defined, that it declares exported constructors names $\mathit{True}$ and $\mathit{False}$, and that it will always incorporated it as part of any use of Plutus Core usage so that the results of these built-ins can be used by programs in case expressions.

%Moving to execution, the computation constructions $\success{M}$, $\failure{}$, $\compbuiltin{n}$, and $\bind{M}{x}{M}$ constitute a first order representation of a reader monad with failure and a particular environment type. Reduction of such terms proceeds as any first order data does. However, such data can also be \emph{executed}, which involves performing actual reader operations as well as failing. We can make an analogy to Haskell's $IO$, where an $IO$ value is just a value, but certain designated names with $IO$ type, in additional to being reduced, are also executed by the run time system. We therefore also define a binary relation \(\executes{E,\delta}{M}{R}\) that specifies when a term $M$ reduces to return value $R$ in some reader environment $E$ and declaration environment $\delta$, as well as a step indexed variant \(\executesIndexed{E,\delta}{M}{n}{R}\). The reader environment $E$ consists of three values, a bytestring \(E_{txhash}\) which is the hash of the host transaction, an integer \(E_{blocknum}\) for the block number of the host block, and an integer \(E_{blocktime}\) for the block time of the host block.




\subfile{figures/PlutusCoreReduction}

\subfile{figures/PlutusCoreBuiltins}





%Note that the success and failure terms are not effectful. That is to say, $\failure{}$ does not throw an exception of any sort. They are merely primitive values that represent computational success and failure. They are analogous to Haskell $\name{Maybe}$ values, except that they cannot be inspected, and all computational control is done via the $bind$ construct.






%\section{Basic Validation Program Structure}

%The basic way that validation is done in Plutus Core is slightly different than in Bitcoin Script. Whereas in Bitcoin Script, a validation is successful if the validating script successfully executes and leads $true$ on the top of the stack, in Plutus Core, we have special data constructs for validation. In particular, the $\success{V}$ and $\failure{}$. Any program which validates a transaction must declare a function \(\decname{\qual{Validator}{validator}}\), while the corresponding program supplied by the redeemer must declare \(\decname{\qual{Redeemer}{redeemer}}\). The declarations of both are combined into a single set of declarations, and these two declared terms are then composed with a bind. The overall validation, therefore, involves reducing the term \[\bind{\decname{\qual{Redeemer}{redeemer}}}{\variable{x}}{\app{\decname{\qual{Validator}{validator}}}{\variable{x}}}\]. If this executes to produce $\ok{V}$ for some $V$, then the transaction is valid, analogous to Bitcoin Script successfully executing and leaving $true$ on the top of stack. On the other hand, if it reduces to $\err{}$, then the transaction is invalid, analogous to Bitcoin Script either leaving $false$ on the top of stack, or failing to execute. The value returned in the success case is irrelevant to validation but may be used for other purposes.




















% trigger a \newpage just before the given reference
% number - used to balance the columns on the last page
% adjust value as needed - may need to be readjusted if
% the document is modified later
%\IEEEtriggeratref{8}
% The "triggered" command can be changed if desired:
%\IEEEtriggercmd{\enlargethispage{-5in}}

% references section

% can use a bibliography generated by BibTeX as a .bbl file
% BibTeX documentation can be easily obtained at:
% http://www.ctan.org/tex-archive/biblio/bibtex/contrib/doc/
% The IEEEtran BibTeX style support page is at:
% http://www.michaelshell.org/tex/ieeetran/bibtex/
%\bibliographystyle{IEEEtran}
% argument is your BibTeX string definitions and bibliography database(s)
%\bibliography{IEEEabrv,../bib/paper}
%
% <OR> manually copy in the resultant .bbl file
% set second argument of \begin to the number of references
% (used to reserve space for the reference number labels box)

\begin{thebibliography}{1}

%\bibitem{pfpl}
%Harper, R. \emph{Practical Foundations for Programming Languages}.

\bibitem{pfenning-bidi}
Pfenning, F. \emph{Lecture Notes on
Bidirectional Type Checking}. 2004. \url{https://www.cs.cmu.edu/~fp/courses/15312-f04/handouts/15-bidirectional.pdf}

\bibitem{christiansen-bidi}
Christiansen, D. R. \emph{Bidirectional Typing Rules: A Tutorial}. \url{http://www.davidchristiansen.dk/tutorials/bidirectional.pdf}

\end{thebibliography}

% biography section
% 
% If you have an EPS/PDF photo (graphicx package needed) extra braces are
% needed around the contents of the optional argument to biography to prevent
% the LaTeX parser from getting confused when it sees the complicated
% \includegraphics command within an optional argument. (You could create
% your own custom macro containing the \includegraphics command to make things
% simpler here.)
%\begin{biography}[{\includegraphics[width=1in,height=1.25in,clip,keepaspectratio]{mshell}}]{Michael Shell}
% or if you just want to reserve a space for a photo:

%\begin{IEEEbiography}[{\includegraphics[width=1in,height=1.25in,clip,keepaspectratio]{picture}}]{John Doe}
%\blindtext
%\end{IEEEbiography}

% You can push biographies down or up by placing
% a \vfill before or after them. The appropriate
% use of \vfill depends on what kind of text is
% on the last page and whether or not the columns
% are being equalized.

%\vfill

% Can be used to pull up biographies so that the bottom of the last one
% is flush with the other column.
%\enlargethispage{-5in}




% that's all folks
\end{document}


