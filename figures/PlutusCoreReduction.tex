\documentclass[../main.tex]{subfiles}

\begin{document}





\begin{figure*}
    \centering
    \[\begin{array}{lrclr}
        \textrm{Type Frame} & f  & ::= & \inFunTLeft{\_}{A}                        & \textrm{left arrow}\\
                            &    &     & \inFunTRight{S}{\_}                       & \textrm{right arrow}\\
                            &    &     & \inAllT{\alpha}{K}{\_}                    & \textrm{all}\\
                            &    &     & \inLamT{\alpha}{K}{\_}                    & \textrm{$\lambda$}\\
                            &    &     & \inAppTLeft{\_}{A}                        & \textrm{left app}\\
                            &    &     & \inAppTRight{S}{\_}                       & \textrm{right app}\\
    \end{array}\]
    
    \caption{Grammar of Type Reduction Frames}
    \label{fig:Plutus_core_type_reduction_frames}
\end{figure*}





\begin{figure*}[t]
    \judgmentdef{\(\typeStep{A}{A'}\)}{Type $A$ reduces in one step to type $A'$}
    
    \begin{prooftree}
        \AxiomC{}
        \UnaryInfC{\(\typeStep{\appT{\lamT{\alpha}{K}{B}}{S}}{\subst{S}{\alpha}{B}}\)}
    \end{prooftree}
    
    \begin{prooftree}
        \AxiomC{\(\typeStep{A}{A'}\)}
        \UnaryInfC{\(\typeStep{\ctxsubst{f}{A}}{\ctxsubst{f}{A'}}\)}
    \end{prooftree}
    
    
    
    \caption{Type Reduction via Contextual Dynamics}
    \label{fig:Plutus_core_type_reduction}
\end{figure*}





\begin{figure*}[t]
    \centering
    \[\begin{array}{lrclr}
        \textrm{Frame} &   &     & \inInstLeftFrame{A}                     & \textrm{left instantiation}\\
                       %&   &     & \inInstRightFrame{V}                    & \textrm{right instantiation}\\
                       %&   &     & \inWrapLeftFrame{\alpha}{M}             & \textrm{left wrap}\\
                       &   &     & \inWrapRightFrame{\alpha}{A}            & \textrm{right wrap}\\
                       &   &     & \inUnwrapFrame{}                        & \textrm{unwrap}\\
                       %&   &     & \inLamLeftFrame{x}{M}                   & \textrm{$\lambda$}\\
                       &   &     & \inAppLeftFrame{M}                      & \textrm{left app}\\
                       &   &     & \inAppRightFrame{V}                     & \textrm{right app}\\
    \end{array}\]
    \caption{Grammar of Reduction Frames}
    \label{fig:Plutus_core_reduction_frames}
\end{figure*}




\begin{figure*}[t]
    \judgmentdef{\(\step{M}{M'}\)}{Term $M$ reduces in one step to term $M'$}
    
    \begin{prooftree}
        \AxiomC{}
        \UnaryInfC{\(\step{\inst{\abs{\alpha}{K}{M}}{S}}{\subst{S}{\alpha}{M}}\)}
    \end{prooftree}
    
    \begin{prooftree}
        \AxiomC{}
        \UnaryInfC{\(\step{\unwrap{\wrap{\alpha}{A}{V}}}{V}\)}
    \end{prooftree}
    
    \begin{prooftree}
        \AxiomC{}
        \UnaryInfC{\(\step{\app{\lam{x}{A}{M}}{V}}{\subst{V}{x}{M}}\)}
    \end{prooftree}
    
    \begin{prooftree}
        \alwaysNoLine
        \AxiomC{$M$ is a fully saturated constant}
        \UnaryInfC{$M$ computes to $V$ according to Fig \ref{fig:Plutus_core_builtins}}
        \alwaysSingleLine
        \UnaryInfC{\(\step{M}{V}\)}
    \end{prooftree}
    
    \red{What does ``fully saturated'' mean?}

    %\begin{prooftree}
    %    \AxiomC{\(\typeStep{A}{A'}\)}
    %    \UnaryInfC{\(\step{\ctxsubst{f}{A}}{\ctxsubst{f}{A'}}\)}
    %\end{prooftree}
    
    \begin{prooftree}
        \AxiomC{\(\step{M}{M'}\)}
        \AxiomC{\(M' = \error{B}\)}
        \RightLabel{\footnotesize\textit{($A$ is the type of the frame, $B$ is the type of its hole)}}
        \BinaryInfC{\(\step{\ctxsubst{f}{M}}{\error{A}}\)}
    \end{prooftree}
    
    \begin{prooftree}
        \AxiomC{\(\step{M}{M'}\)}
        \AxiomC{\(M' \not= \error{B}\)}
        \BinaryInfC{\(\step{\ctxsubst{f}{M}}{\ctxsubst{f}{M'}}\)}
    \end{prooftree}
    
    
    
    
    
    
    \caption{Reduction via Contextual Dynamics}
    \label{fig:Plutus_core_reduction}
\end{figure*}





\begin{figure*}
    \centering
    \[\begin{array}{lrclr}
        \textrm{Stack} & s      & ::= & f^*                               & \textrm{stacks}\\
        \textrm{State} & \sigma & ::= & \ckforward{s}{M}                  & \textrm{computing a term}\\
                       %&        &     & \ckforward{s}{A}                  & \textrm{computing a type}\\
                       &        &     & \ckbackward{s}{V}                 & \textrm{returning a term value}\\
                       %&        &     & \ckbackward{s}{S}                 & \textrm{returning a type value}\\
                       &        &     & \ckerror{}                        & \textrm{throwing an error}
    \end{array}\]

\red{In Figure 6, $f$ is defined to be a Type Frame.  I think we actually want plain
Frames here, which are defined in Figure 8 without any name.  Note that $f$ reappears later,
in Figure 16 where it's got something to do with erasure.}
    
    \caption{Grammar of CK Machine States}
    \label{fig:Plutus_core_ck_frames}
\end{figure*}



\begin{figure*}[t]
    \judgmentdef{\(\cksteps{\sigma}{\sigma'}\)}{Machine state $\sigma$ transitions in one step to machine state $\sigma'$}
    
    \begin{align*}
        &\cksteps{\ckforward{s}{\abs{\alpha}{K}{M}}}{\ckbackward{s}{\abs{\alpha}{K}{M}}}\\
        &\cksteps{\ckforward{s}{\inst{M}{A}}}{\ckforward{s, \inInstLeftFrame{A}}{M}}\\
        &\cksteps{\ckforward{s}{\wrap{\alpha}{A}{M}}}{\ckforward{s, \inWrapRightFrame{\alpha}{A}}{M}}\\
        &\cksteps{\ckforward{s}{\unwrap{M}}}{\ckforward{s, \inUnwrapFrame{}}{M}}\\
        &\cksteps{\ckforward{s}{\lam{x}{A}{M}}}{\ckbackward{s}{\lam{x}{A}{M}}}\\
        &\cksteps{\ckforward{s}{\app{M}{N}}}{\ckforward{s, \inAppLeftFrame{N}}{M}}\\
        &\cksteps{\ckforward{s}{\con{cn}}}{\ckbackward{s}{\con{cn}}}\\
        &\cksteps{\ckforward{s}{\error{S}}}{\ckerror{}}\\
        &\cksteps{\ckbackward{s, \inInstLeftFrame{A}}{\abs{\alpha}{K}{M}}}{\ckforward{s}{M}}\\
        &\cksteps{\ckbackward{s, \inInstLeftFrame{A}}{M}}{\ckbackward{s}{V}} \quad \textit{fully saturated constant} \quad\mbox{\red{Where does V come from?}}\\
        &\cksteps{\ckbackward{s, \inInstLeftFrame{A}}{M}}{\ckbackward{s}{\inst{M}{A}}} \quad \textit{partially saturated constant}\\
        &\cksteps{\ckbackward{s, \inWrapRightFrame{\alpha}{A}}{V}}{\ckbackward{s}{\wrap{\alpha}{A}{V}}}\\
        &\cksteps{\ckbackward{s, \inUnwrapFrame{}}{\wrap{\alpha}{A}{V}}}{\ckbackward{s}{V}}\\
        &\cksteps{\ckbackward{s, \inAppLeftFrame{N}}{V}}{\ckforward{s, \inAppRightFrame{V}}{N}}\\
        &\cksteps{\ckbackward{s, \inAppRightFrame{\lam{x}{A}{M}}}{V}}{\ckforward{s}{\subst{V}{x}{M}}}\\
        &\cksteps{\ckbackward{s,\inAppRightFrame{M}}{N}}{\ckbackward{s}{V}} \quad \textit{fully saturated constant} \quad\mbox{\red{Where does V come from?}}\\
        &\cksteps{\ckbackward{s,\inAppRightFrame{M}}{N}}{\ckbackward{s}{\app{M}{N}}} \quad \textit{partially saturated constant}\\
    \end{align*}

    \caption{CK Machine}
    \label{fig:Plutus_core_ck_machine}
%%%%%%%%%%%%%
\vskip5mm
\red{I couldn't see any definition of ``fully saturated'' and ``partially saturated'' constants. Also,
why not just ``saturated and ``unsaturated''?}
\vskip5mm
\red{The machine has states of the form $s \triangleright M$ and $s \triangleleft V$
where $M$ is a term and $V$ is a value (I think), but there are a number of inconsistencies with the grammar.
For example:
\begin{itemize}
\item There's a transition for the state $s \triangleright \{M A\}$, but according to Figure 3 
 $\{M A\}$ isn't a term; there is a term $\{M S\}$, but this requires a type value (normalised type?)
$S$, and $A$ denotes a type.  In fact, all of the transitions listed above involve $A$s, but terms 
in the grammar all contain $S$s.
%
\item If $x$ is a variable name then $x$ is a term, but there's no transition for this.
%
\item There's a state $s, [M \_] \triangleleft N$.  We're supposed to have a $V$ on the right of the
$\triangleleft$, but I think $N$ is a term instead.
Similarly, I don't think $[M N]$ on the right of the last transition is a $V$.
%
\item Hmmm.  We have an implementation of this, so it must make sense to someone...
\end{itemize}
}
%%%%%%%%%%%%%
\end{figure*}



\end{document}
